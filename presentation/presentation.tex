\documentclass{beamer}

\usepackage{textcomp}
\usepackage{xcolor}
\usepackage{listings}
\usepackage{matlab-prettifier}
\usepackage[T1]{fontenc}
\usepackage{multimedia}
\usepackage{hyperref}

\usetheme{Boadilla}

\makeatletter
\def\blfootnote{\gdef\@thefnmark{}\@footnotetext}
\makeatother
\urlstyle{sf}

\title{Xolotl}
\subtitle{A fast and flexible neuronal simulator}
\author{Alec Hoyland}
\institute[CSN]{Center for Systems Neuroscience}

\begin{document}

%% title page

\begin{frame}
  \titlepage
\end{frame}


\section{What is xolotl?}

\begin{frame}
  \frametitle{Structure of Talk}

  \begin{columns}
    \column{0.5\textwidth}

    "What" more than "Why and How"

    \begin{enumerate}
      \item What is xolotl?
      \item Features
      \item Demonstrations
      \begin{enumerate}
        \item My first neuron
        \item My first network
        \item Demos \& interactive demos
      \end{enumerate}
    \end{enumerate}

    \column{0.5\textwidth}

    \includegraphics[width=\textwidth]{gfx/xolotl.png}

\end{columns}

  \blfootnote{Contact me at: ahoyland@bu.edu}

\end{frame}

\subsection{Rationale}

%% Design Principles

\begin{frame}
  \frametitle{Design Principles}

  \begin{columns}
    \column{0.5\textwidth}

    Xolotl should be

    \begin{itemize}
      \item fast
      \item easy-to-use
      \item well-documented
      \item hackable and extensible
      \item auditable
    \end{itemize}

    \column{0.5\textwidth}

    \includegraphics[width=\textwidth]{gfx/fast.png}
    \centering

    \includegraphics[width=\textwidth]{gfx/documentation.png}
    \centering

  \end{columns}

\end{frame}

\subsection{Design}

%% How Xolotl Works

\begin{frame}
  \frametitle{How xolotl works}

  \begin{figure}
    \includegraphics[width=\textwidth]{gfx/fig1.jpg}
    \caption{Model of A \& B represented in code C which produces D-I.}
  \end{figure}

\end{frame}

%% Anatomy of a Model

\begin{frame}[fragile]
  \frametitle{Anatomy of a model}

  \begin{columns}

    \column{0.35\textwidth}
    Types of Components:

    \begin{itemize}
      \item Compartments
      \begin{itemize}
        \item Mechanisms
      \end{itemize}
      \begin{itemize}
        \item Conductances
        \begin{itemize}
          \item Mechanisms
        \end{itemize}
        \item Synapses
        \begin{itemize}
          \item Mechanisms
        \end{itemize}
      \end{itemize}
    \end{itemize}


    \begin{figure}
      \includegraphics[width=\textwidth]{gfx/liu.png}
      \centering
      \caption{100+ components are a searchable, indexed feature of the language.}
    \end{figure}

    \column{0.65\textwidth}

    Code to generate an HH model with constant injected current:

    \resizebox{\textwidth}{!}{\lstinputlisting[style=Matlab-editor,breaklines=false]{code/HH.m}}

  \end{columns}

\end{frame}

\subsection{Features}

%% Cool Features

\begin{frame}{Embedded Animation}
  \frametitle{Cool features}

    \begin{itemize}
      \item \texttt{puppeteer}: real-time parameter manipulation
      \item \texttt{xgrid}: parallel simulation across a distributed network
      \item \texttt{xfit}: parameter optimization using particle swarm and genetic algorithms
      \item \texttt{xtools}: spike counting and data analysis
      \item model hashing and snapshotting
      \item control over input and output (clamping, full state matrix)
      \item automatic component generation from MATLAB
      \item hyperlinking and tab-completion in the console
      \item multiple solvers, look-up table caching
    \end{itemize}

\end{frame}

%% Coming Soon

\begin{frame}{Embedded Animation}
  \frametitle{Coming Soon}

    \begin{itemize}
      \item multi-threading of a single simulation
      \item server-side compilation / stand-alone integration
      \item (multi-compartment) server-side GPU computation of Hines matrices
      \item new compartment types (including low-dimensional models)
      \item universal support for Runge-Kutta integration schemes
      \item adaptive time-step solvers (quadrature)
      \item robust front-end unit support
      \item Julia front-end (compatible with Python, etc.)
    \end{itemize}

\end{frame}

%% Pretty Plots

\begin{frame}
  \frametitle{Real-time manipulation}

    \begin{center}
      \movie[width=\textwidth, height=0.3\textwidth, loop]{}{gfx/manipulate-neuron.mp4}
      \caption{Real-time parameter manipulation. Any numerical xolotl property can be manipulated.}
    \end{center}

\end{frame}

\begin{frame}
  \frametitle{\texttt{xfit}: Parameter optimization}

  \begin{columns}
    \column{0.5\textwidth}
    Optimized for:

    \begin{enumerate}
      \item Slow-wave troughs at -70 mV.
      \item Slow-wave peaks at -40 mV.
      \item Spike downswing ends above slow wave trough.
      \item Burst frequency of 0.5 Hz.
      \item Duty cycle of 0.3.
    \end{enumerate}

    \column{0.5\textwidth}
    \begin{figure}
      \includegraphics[width=\textwidth]{gfx/xfit.png}
      \centering
      \caption{Fit of 8-conductance model (left to right). PSO \#2 shown.}
    \end{figure}
  \end{columns}

\end{frame}

\section{Demonstrations}

\subsection{Installing}

%% Getting Started

\begin{frame}[fragile]
  \frametitle{Installing}

  Acquiring the MATLAB toolbox
  \medskip

  \begin{enumerate}
    \item Go to \url{https://github.com/sg-s/xolotl/releases/latest}
    \item Download \texttt{xolotl.mltbox}
    \item Find the file in \texttt{Downloads} and drag it onto your \texttt{MATLAB} workspace
          This will install \texttt{xolotl}
  \end{enumerate}

  \blfootnote{\url{https://xolotl.readthedocs.io/en/master/tutorials/start-here/}}

\end{frame}

%% Setup in MATLAB

\begin{frame}[fragile]

  Run the following commands in MATLAB. You should see this plot.

  \resizebox{0.5\textwidth}{!}{\lstinputlisting[style=Matlab-editor,breaklines=false]{code/setup.m}}
  \medskip

  \begin{figure}
    \includegraphics[width=\textwidth]{gfx/demo_bursting_neuron.png}
  \end{figure}

  \blfootnote{\url{https://xolotl.readthedocs.io/en/master/tutorials/built-in-demos/}}

\end{frame}

%% Demos

\subsection{Demonstrations}

\begin{frame}
  \frametitle{Demonstrations}

  Your first neuron

  \medskip

  \url{https://xolotl.readthedocs.io/en/master/tutorials/first-neuron/}

  \medskip
  Your first network
  \medskip

  \url{https://xolotl.readthedocs.io/en/master/tutorials/first-network/}

  \medskip
  All demos
  \medskip

  \url{https://xolotl.readthedocs.io/en/master/tutorials/built-in-demos/}\medskip

\end{frame}


\end{document}
 
